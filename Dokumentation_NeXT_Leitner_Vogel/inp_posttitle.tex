\chapter*{Eidesstattliche Erklärung}
Ich erkläre an Eides statt, dass ich die vorliegende Diplomarbeit selbst verfasst und keine anderen als die angeführten Behelfe verwendet habe. Alle Stellen, die wörtlich oder inhaltlich den angegebenen Quellen entnommen wurden, sind als solche kenntlich gemacht.
Ich bin damit einverstanden, dass meine Arbeit öffentlich zugänglich gemacht wird.

\vspace{1cm}
\begin{tabular}{c c c}
	& \hspace{4cm} & \\\cline{1-1}
	Ort, Datum & & \\
	\vspace{2cm}
	& & \\\cline{1-1}\cline{3-3}
	Michael Leitner & & Lukas Vogel \\ 
	\vspace{2cm}

\end{tabular}

\chapter*{Abnahmeerklärung}
Hiermit bestätigt der Auftraggeber, dass das übergebene Produkt dieser Diplomarbeit den dokumentierten Vorgaben entspricht. Des Weiteren verzichtet der Auftraggeber auf unentgeltliche Wartung und Weiterentwicklung des Produktes durch die Projektmitglieder bzw. die Schule.

\vspace{1cm}
\begin{tabular}{c}
	\\\cline{1-1}
	Ort, Datum\\
	\vspace{2cm}
	\\\cline{1-1}
	Auftraggeber
\end{tabular}	

\chapter*{Vorwort}
z. B. Hinweise, wie das bearbeitete Thema gefunden wurde oder Dank für die Betreuung (Kooperationspartner/in, Betreuer/innen, Sponsoren) etc.


\chapter*{Abstract (Deutsch)}
\begin{table}
	\caption{Abstract Deutsch}
\begin{tabular}{|l|l|}
	\hline 
	Thema: &  Prototyp: NeXt\\ 
	\hline
	 Name der Verfasser: & Michael Leitner \& Lukas Vogel  \\ 
	\hline 
	Jahrgang: & ------ \\
	\hline
	 Schuljahr: & 2017/18 \\
	\hline 
	Kooperationspartner: & ClockStone Softwareentwicklung GmbH\\
	\hline
\end{tabular}
\\
\\
\begin{tabular}{|l|l|}
	\hline
	Aufgabenstellung & ---- \\
	\hline
\end{tabular} 
\\
\\
\begin{tabular}{|l|l|}
	\hline
	Realisierung: & ----\\
	\hline
\end{tabular}
\\
\\
\begin{tabular}{|l|l|}
	\hline
	Ergebnisse: & ----\\
	\hline
\end{tabular}
\end{table}
Das Ziel der vorliegenden Diplomarbeit war es, ein Computerspiel des Jump and Run und des Puzzle Genres zu entwickeln. Für diesen Zweck wurde die Unity-Engine verwendet und die vom selben Hersteller mitgelieferte Entwicklungsumgebung welche auf der Programmiersprache C\# basiert. In der Diplomarbeit werden wesentlichen Source-Code Elemente aus dem Spiel veranschaulicht und auch erklärt und somit die Entwicklung des Projekts gezeigt. Diese Dokumentation bietet einen Einblick in die Spielentwicklung und kann somit für jeden eine Hilfe sein der sich mit diesem Thema beschäftigt.

\chapter*{Abstract (Englisch)}
(ca. ½ bis max. 2 Seiten)