

\chapter*{Eidesstattliche Erklärung}
Ich erkläre an Eides statt, dass ich die vorliegende Diplomarbeit selbst verfasst und keine anderen als die angeführten Behelfe verwendet habe. Alle Stellen, die wörtlich oder inhaltlich den angegebenen Quellen entnommen wurden, sind als solche kenntlich gemacht.
Ich bin damit einverstanden, dass meine Arbeit öffentlich zugänglich gemacht wird.

\vspace{1cm}
\begin{tabular}{c c c}
	& \hspace{4cm} & \\\cline{1-1}
	Ort, Datum & & \\
	\vspace{2cm}
	& & \\\cline{1-1}\cline{3-3}
	Michael Leitner & & Lukas Vogel \\ 
	\vspace{2cm}

\end{tabular}

\chapter*{Abnahmeerklärung}
Hiermit bestätigt der Auftraggeber, dass das übergebene Produkt dieser Diplomarbeit den dokumentierten Vorgaben entspricht. Des Weiteren verzichtet der Auftraggeber auf unentgeltliche Wartung und Weiterentwicklung des Produktes durch die Projektmitglieder bzw. die Schule.

\vspace{1cm}
\begin{tabular}{c}
	\\\cline{1-1}
	Ort, Datum\\
	\vspace{2cm}
	\\\cline{1-1}
	Auftraggeber
\end{tabular}	

\chapter*{Vorwort}
Viele Informatiker spielen Computerspiele und so manch einer wünscht es sich auch eines selbst zu entwickeln. So ging es auch uns, glücklicherweise konnten wir diesen Wunsch durch diese Diplomarbeit verwirklichen. Wir entwickelten im Zuge dieses Projekts einen Prototyp für ein Spiel und lernten dabei einiges über die Spielentwicklung und welche Arbeiten gemacht werden müssen um ein Projekt dieser Größe zu bewerkstelligen.

Herzlich bedanken möchten wir uns bei unserem Betreuer Claudio Landerer und er Firma ClockStone, Innsbruck, die uns bei der Umsetzung unseres Projektes tatkräftig unterstützt haben. 


\chapter*{Abstract (Deutsch)}
\def \currentAuthor {Lukas Vogel}
\begin{table}[H]
	\caption{Abstract Deutsch}
	\renewcommand{\arraystretch}{1.5}
\begin{tabular}{|p{4cm}|p{10cm}|}
	\hline 
	Thema: &  Prototyp: NeXt\\ 
	\hline
	 Name der Verfasser: & Michael Leitner \& Lukas Vogel  \\ 
	\hline 
	Jahrgang: & 2016/17 \\
	\hline
	 Schuljahr: & 2017/18 \\
	\hline 
	Kooperationspartner: & ClockStone Softwareentwicklung GmbH\\
	\hline
\end{tabular}
\ \\
\ \\
\begin{tabular}{|p{4cm}|p{10cm}|}
	\hline
	Aufgabenstellung: & Das Ziel dieser Diplomarbeit war es ein Prototyp für das Computerspiel NeXt, welches Elemente aus dem Puzzle und dem Jump and Run Genre beinhaltet, zu entwickeln. Besonders Hervorzuheben ist das Einbauen des Time-Rift-Prinzips.   \\
	\hline
\end{tabular} 
\ \\
\ \\
\begin{tabular}{|p{4cm}|p{10cm}|}
	\hline
	Realisierung: & Das Spiel wurde mit der Unity-Engine, welche auf der Programmierspracha C\# basiert und mit der vom selben Hersteller mitgelieferten Entwicklungsumgebung verwirklicht. Für die Dokumentation verwendeten wir LaTeX in Kombination mit TeXstudio. Es kamen auch viele Techniken des Projektmanagements zur Anwendung.\\
	\hline
\end{tabular}
\ \\
\ \\
\begin{tabular}{|p{4cm}|p{10cm}|}
	\hline
	Ergebnisse: & Als Resultat dieser Diplomarbeit entstand ein Spielbares Computerspiel welches die oben genannten Ziele erfüllt. Das Spiel hat 4 Level welche das Spielprinzip gut zur Geltung bringen. Die Dokumentation wurde erfolgreich Abgeschlossen und beschreibt das gesamte Projekt. \\
	\hline
\end{tabular}
\end{table}
\chapter*{Abstract (Englisch)}
(ca. ½ bis max. 2 Seiten)